% Created 2015-03-10 Tue 14:45
\documentclass[11pt]{article}
\usepackage[utf8]{inputenc}
\usepackage[T1]{fontenc}
\usepackage{fixltx2e}
\usepackage{graphicx}
\usepackage{longtable}
\usepackage{float}
\usepackage{wrapfig}
\usepackage{rotating}
\usepackage[normalem]{ulem}
\usepackage{amsmath}
\usepackage{textcomp}
\usepackage{marvosym}
\usepackage{wasysym}
\usepackage{amssymb}
\usepackage{hyperref}
\tolerance=1000
\usepackage{fullpage}
\author{Rex}
\date{\today}
\title{CSE-605 Checkpoint 1}
\hypersetup{
  pdfkeywords={},
  pdfsubject={},
  pdfcreator={Emacs 25.0.50.16 (Org mode 8.2.10)}}
\begin{document}

\maketitle
\tableofcontents


\section{{\bfseries\sffamily TODO} Introduction}
\label{sec-1}
We will measure the predictability of Android.
How far ordinary Android system between Real-time system.

\section{{\bfseries\sffamily TODO} Android Components}
\label{sec-2}
\begin{itemize}
\item Intents and Intent Filters
We will evaluate intent delivery mechanism of Android.
The intents will appear for almost all multiple processes experiments.

\item Activities
\item App Widgets

We will not directly evaluate activities and app widgets because they're only related with UI,
it's hard to produce convincible result because there're too many elements out of control,
like the GPU power, the screen resolution.

\item Services

We will not directly evalute services as well. The reason is same as activities.
\item Content Providers

We will evalute content providers related \textbf{Garbage Collector}, \textbf{Synchronization},
and \textbf{Scheduler}. Because it's shared data mechanism Android provied.
\end{itemize}


\begin{itemize}
\item Processes and Threads
We will evaluate processes and threads directly.
They're both our targets and mechanisms to use.
\end{itemize}

\section{Intents/Intent Filters}
\label{sec-3}
\subsection{Single Process}
\label{sec-3-1}
X, one process doesn't make sense?

\subsection{Multiple Processes}
\label{sec-3-2}
We can use process(es) to generate bunch of intents, then use other
process(es) to receive the intents.
So we can evaluate the order and time of intent delivery.
It can provide some pressure for \textbf{Garbage Collector}, \textbf{Synchronization}, and \textbf{Scheduler}.

\subsubsection{Garbage Collector}
\label{sec-3-2-1}
We can use one sender and one receiver to test Garbage Collector,
we can associate different size objects with intents.
Then the receiver decide how to release those objects.
So we can evaluate how garbage collector works:
\begin{itemize}
\item frequency of grabage collection and memory pressure
\item running time of grabage collection and memory pressure
\end{itemize}

The memory pressure should contains different types:
\begin{center}
\begin{tabular}{l|lll}
 & big objects & medium objects & small objects\\
\hline
long live time &  &  & \\
short live time &  &  & \\
\end{tabular}
\end{center}

\subsubsection{Synchronization}
\label{sec-3-2-2}
This may need other Android components.
Because we can not pass an object as extra of an intent, we need serialize the object first.
So there no directly synchronized mechanism between intent and receiver,
but we can pass some meta data to let receivers use something need synchronization like \textbf{Content Provider}.

\subsubsection{Scheduler}
\label{sec-3-2-3}
We can use multiple processes to generate intents for multiple receivers.
So the scheduler will get pressure, then we use the order of intent delivery
to evaluate scheduler and intent delivery mechanism.

\section{{\bfseries\sffamily TODO} Content Providers}
\label{sec-4}
\section{{\bfseries\sffamily TODO} Processes/Threads}
\label{sec-5}
\section{Parcelable/Serializable}
\label{sec-6}
% Emacs 25.0.50.16 (Org mode 8.2.10)
\end{document}